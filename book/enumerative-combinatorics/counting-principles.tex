% !TeX root = ../../infdesc.tex
\section{Counting principles}
\secbegin{secCountingPrinciples}

\index{counting|(}

In \Cref{secFiniteSets} we were interested in establishing conditions under which a set is finite, and proving that we may perform certain operations on finite sets---such as unions and cartesian products---without losing the property of finiteness.

In this section, our attention turns to the task of finding the size of a set that is known to be finite. This process is called \textit{counting} and is at the core of the mathematical field of combinatorics.

\subsection*{Binomials and factorials revisited}
We defined binomial coefficients $\binom{n}{k}$ and factorials $n!$ \textit{recursively} in \Cref{chMathematicalInduction}, and proved elementary facts about them by induction. We will now re-define them \textit{combinatorially}---that is, we give them meaning in terms of sizes of particular finite sets. We will prove that the combinatorial and recursive definitions are equivalent, and prove facts about them using combinatorial arguments.

The reasons for doing so are manifold. The combinatorial definitions allow us to reason about binomials and factorials with direct reference to descriptions of finite sets, which will be particularly useful when we prove identities about them using
\textit{double counting}.
% \textit{counting in two ways}.
Moreover, the combinatorial definitions remove the seemingly arbitrary nature of the recursive definitions---for example, they provide a reason why it makes sense to define $0!=1$ and $\binom{0}{0}=1$.

\begin{definition}
\label{defXChoosek}
\nindex{X choose k}{$\binom{X}{k}$}{$k$-element subsets}
\index{subset!$k$-element subset}
Let $X$ be a set and let $k \in \mathbb{N}$. A $k$-\textbf{element subset} of $X$ is a subset $U \subseteq X$ such that $|U|=k$. The set of all $k$-element subsets of $X$ is denoted $\binom{X}{k}$ (read: `$X$ choose $k$') \inlatex{binom\{X\}\{k\}}\lindexmmc{binom}{$\binom{n}{k}$}.
\end{definition}

Intuitively, $\binom{X}{k}$ is the set of ways of picking $k$ elements from $X$, without repetitions, in such a way that order doesn't matter. (If order mattered, the elements would be \textit{sequences} instead of \textit{subsets}.)

\begin{example} \label{exFinSubsetsOf4}
We find $\binom{[4]}{k}$ for all $k \in \mathbb{N}$.
\begin{itemize}
\item $\binom{[4]}{0} = \{\varnothing\}$ since the only set with $0$ elements is $\varnothing$;
\item $\binom{[4]}{1} = \{ \{0\}, \{1\}, \{2\}, \{3\} \}$;
\item $\binom{[4]}{2} = \{ \{0,1\}, \{0,2\}, \{0,3\}, \{1,2\}, \{1,3\}, \{2,3\} \}$;
\item $\binom{[4]}{3} = \{ \{0,1,2\}, \{0,1,3\}, \{0,2,3\}, \{1,2,3\} \}$;
\item $\binom{[4]}{4} = \{ \{ 0,1,2,3 \} \}$;
\item If $k \ge 5$ then $\binom{[4]}{k} = \varnothing$, since by \Cref{exSubsetOfFiniteSetIsFinite}, no subset of $[4]$ can have more than $4$ elements.
\end{itemize}
\end{example}

\begin{proposition}
\label{propUnionOfFinSubsetsEqPowerSet}
If $X$ is a finite set, then $\mathcal{P}(X) = \bigcup_{k \le |X|} \binom{X}{k}$.
\end{proposition}
\begin{cproof}
Let $U \subseteq X$. By \Cref{exSubsetOfFiniteSetIsFinite}, $U$ is finite and $|U| \le |X|$. Thus $U \in \binom{X}{|U|}$, and hence $U \in \bigcup_{k \le |X|} \binom{X}{k}$. This proves that $\mathcal{P}(X) \subseteq \bigcup_{k \le |X|} \binom{X}{k}$.

The fact that $\bigcup_{k \le |X|} \binom{X}{k} \subseteq \mathcal{P}(X)$ is immediate, since elements of $\binom{X}{k}$ are defined to be subsets of $X$, and hence elements of $\mathcal{P}(X)$.
\end{cproof}

\begin{definition}
\label{defBinomialCoefficient}
\index{binomial coefficient}
Let $n,k \in \mathbb{N}$. Denote by $\binom{n}{k}$\nindex{nchoosek}{$\binom{n}{k}$}{binomial coefficient} (read: `$n$ choose $k$') \inlatex{binom\{n\}\{k\}} the number of $k$-element subsets of $[n]$. That is, we define $\binom{n}{k} = \left|\binom{[n]}{k}\right|$. The numbers $\binom{n}{k}$ are called \textbf{binomial coefficients}.
\end{definition}

Some authors use the notation ${}_n\mathrm{C}_k$ or ${}^n\mathrm{C}_k$ instead of $\binom{n}{k}$. We avoid this, as it is unnecessarily clunky.

Intuitively, $\binom{n}{k}$ is the number of ways of selecting $k$ things from $n$, without repetitions, in such a way that order doesn't matter.

The value behind this notation is that it allows us to express huge numbers in a concise and meaningful way. For example,
\[ \binom{4000}{11} = 103\;640\;000\;280\;154\;258\;645\;590\;429\;564\;000 \]
Although these two numbers are equal, their \textit{expressions} are very different; the expression on the left is meaningful, but the expression on the right is completely meaningless out of context.

\begin{writingtip}
When expressing the sizes of finite sets described combinatorially, it is best to \textit{avoid} evaluating the expression; that is, leave in the powers, products, sums, binomial coefficients and factorials! The reason for this is that performing the calculations takes the meaning away from the expressions.
\end{writingtip}

\begin{example} \label{exBinomCalc4}
In \Cref{exFinSubsetsOf4} we proved that:
\[ \binom{4}{0} = 1,\ \binom{4}{1} = 4,\ \binom{4}{2} = 6,\ \binom{4}{3} = 4,\ \binom{4}{4} = 1 \]
and that $\binom{4}{k} = 0$ for all $k \ge 5$.
\end{example}

\begin{exercise}
Fix $n \in \mathbb{N}$. Prove that $\binom{n}{0} = 1$, $\binom{n}{1} = n$ and $\binom{n}{n} = 1$.
\end{exercise}

\begin{definition}
\label{defPermutationPreliminary}
\index{permutation}
Let $X$ be a set. A \textbf{permutation} of $X$ is a bijection $X \to X$.
\end{definition}

\begin{example}
\label{exPermutationsOfBracket3}
There are six permutations of the set $[3]$. Representing each bijection $f : [3] \to [3]$ by the ordered triple $(f(0),f(1),f(2))$ of its values, these permutations are thus given by
\[ (0,1,2),\ (0,2,1),\ (1,0,2),\ (1,2,0),\ (2,0,1),\ (2,1,0) \]
For example, $(1,2,0)$ represents the permutation $f : [3] \to [3]$ defined by $f(0)=1$, $f(1)=2$ and $f(2)=0$.
\end{example}

\begin{exercise}
List all the permutations of the set $[4]$.
\end{exercise}

\begin{definition}
\label{defFactorial}
\index{factorial}
Let $n \in \mathbb{N}$. Denote by $n!$\nindex{nfactorial}{$n"!$}{factorial} (read: `$n$ factorial') the number of permutations of $[n]$. That is,
\[ n! = |\{ f : [n] \to [n] \mid f \text{ is a bijection} \}| \]
The numbers $n!$ are called \textbf{factorials}.
\end{definition}

\begin{example}
\Cref{exPermutationsOfBracket3} shows that $3!=6$.
\end{example}

The result of the following exercise allows us to interpret factorials in different ways than just permutations of sets.

\begin{exercise}[Other characterisations of the factorial]
\label{exOtherCharacterisationsOfFactorial}
Let $n \in \mathbb{N}$, and let $X$ and $Y$ be finite sets with $|X|=|Y|=n$.

Prove that the following sets all have size $n!$:
\begin{enumerate}[(a)]
\item The set of all $n$-tuples $(x_0,x_1,\dots,x_{n-1}) \in X^n$ such that each element of $X$ appears exactly once in the list;
\item The set of all permutations of $X$;
\item The set of all enumerations of $X$ (that is, bijections $[n] \to X$); and
\item The set of all bijections $X \to Y$.
\end{enumerate}
\end{exercise}

\subsection*{Products and procedures}

We saw in \Cref{propProductOfFiniteSetsIsFinite} that, given two finite sets $X$ and $Y$, the product $X \times Y$ is finite. We also found a formula for its size, namely $|X \times Y| = |X| \cdot |Y|$. The \textit{multiplication principle} (\Cref{thmMultiplicationPrincipleGeneral}) generalises this formula to products that may contain any finite number of sets, not just two.

\begin{lemma}
\label{lemMultiplicationPrincipleIndependent}
\index{counting principle!multiplication principle}
\index{multiplication principle}
\index{rule of product}
Let $\{ X_1, \dots, X_n \}$ be a family of finite sets, with $n \ge 1$. Then $\prod_{i=1}^n X_i$ is finite, and
\[ \left| \prod_{i=1}^n X_i \right| = |X_1| \cdot |X_2| \cdot \cdots \cdot |X_n| \]
\end{lemma}

\begin{cproof}
We proceed by induction on $n$.
\begin{itemize}
\item (\textbf{Base case}) When $n=1$, an element of $\prod_{i=1}^1 X_i$ is `officially' a sequence $(x_1)$ with $x_1 \in X_1$. This is the same as an element of $X_1$, in the sense that the assignments $(x_1) \mapsto x_1$ and $x_1 \mapsto (x_1)$ define mutually inverse (hence bijective) functions between $\prod_{i=1}^1 X_i$ and $X_1$, and so
\[ \left| \prod_{i=1}^1 X_i \right| = |X_1| \]
\item (\textbf{Induction step}) Fix $n \in \mathbb{N}$, and suppose that
\[ \displaystyle \left| \prod_{i=1}^n X_i \right| = |X_1| \cdot |X_2| \cdot \cdots \cdot |X_n| \]
for all sets $X_i$ for $i \in [n]$. This is our induction hypothesis.

Now let $X_1, \dots, X_n, X_{n+1}$ be sets. We define a function
\[ F : \prod_{i=1}^{n+1} X_i \to \left(\prod_{i=1}^n X_i\right) \times X_{n+1} \]
by letting $F((x_1, \dots, x_n, x_{n+1})) = ((x_1, \dots, x_n), x_{n+1})$. It is again easy to check that $F$ is a bijection, and hence
\[ \left| \prod_{i=1}^{n+1} X_i \right| = \left| \prod_{i=1}^n X_i \right| \cdot |X_{n+1}| \]
by \Cref{propProductOfFiniteSetsIsFinite}. Applying the induction hypothesis, we obtain the desired result, namely
\[ \left| \prod_{i=1}^{n+1} X_i \right| = |X_1| \cdot |X_2| \cdot \cdots \cdot |X_n| \cdot |X_{n+1}| \]
\end{itemize}
By induction, we're done.
\end{cproof}

\Cref{lemMultiplicationPrincipleIndependent} gives rise to a useful strategy for computing the size of a finite set $X$---see \Cref{thmMultiplicationPrincipleIndependent}. Intuitively, by devising a step-by-step procedure for specifying an element of $X$, we are constructing a cartesian product $\prod_{k=1}^n X_k$, where $X_k$ is the set of choices to be made in the $k^{\text{th}}$ step. This establishes a bijection $\prod_{k=1}^n X_k \to X$, which by bijective proof (\Cref{strComparingSizesOfFiniteSets}\ref{strBijectiveProof}) lets us compute $|X|$ as the product of the numbers of choices that can be made in each step.

\begin{definition}
\label{defSpecificationProcedure}
\index{specification procedure}
A \textbf{specification procedure} for a set $X$ is a sequence of $r$ steps, for some $r \in \mathbb{N}$, such that:
\begin{enumerate}[(i)]
\item In each step, one of a predetermined set of choices is made---which choices are available may in general depend on the choices made in the previous steps;
\item Each sequence of choices determines an element of $X$; and
\item Each element of $X$ is uniquely determined by a unique sequence of choices.
\end{enumerate}  
\end{definition}

\begin{example}
Let $n \in \mathbb{N}$. A specification procedure for $\mathcal{P}([n])$ is as follows. There are $n$ steps, numbered Step $0$ through Step $n-1$. To specify an element $A \in \mathcal{P}([n])$ (that is, a subset $A \subseteq [n]$), for each $k \in [n]$, we decide in Step $k$ whether $k \in A$ or $k \not\in A$.
\end{example}

In the context of \Cref{defSpecificationProcedure}, letting $S$ be the set of all sequences of choices, condition (ii) is equivalent to the existence of a function $S \to X$, which sends a sequence of choices to the element of $X$ determined by those choices, and condition (iii) is equivalent to this function being a bijection. So if $S$ is finite, then so is $X$, and $|X| = |S|$.

If, for each $k \in [r]$, the available choices in Step $k$ are independent of those made in other steps, this means that $S = \displaystyle \prod_{k \in [r]} S_k$, where $S_k$ is the set of choices that are available in Step $k$.

Therefore we can apply \Cref{lemMultiplicationPrincipleIndependent} to obtain the following theorem:

\begin{theorem}[Multiplication principle {(independent version)}]
\label{thmMultiplicationPrincipleIndependent}
Let $X$ be a set, and suppose that there is a specification procedure for $X$ such that in each step, there are finitely many choices available, and the choices are independent of those made in other steps. Then $X$ is finite.

Moreover, letting $r \in \mathbb{N}$ be the number of steps, and letting $m_k$ for $k \in [r]$ be the numbers of choices available in the respective steps, we have $|X| = \prod_{k \in [r]} m_k$. \qed
\end{theorem}

\begin{example}
You go to an ice cream stand selling the following flavours:
\begin{center} vanilla, strawberry, chocolate, rum and raisin, mint choc chip, toffee crunch \end{center}
You can have your ice cream in a tub, a regular cone or a \textit{choco-cone}. You can have one, two or three scoops. We will compute how many options you have.

To select an ice cream, you follow the following procedure:
\begin{itemize}
\item \textbf{Step 1.} Choose a flavour. There are $6$ ways to do this.
\item \textbf{Step 2.} Choose whether you'd like it in a tub, regular cone or choco-cone. There are $3$ ways to do this.
\item \textbf{Step 3.} Choose how many scoops you'd like. There are $3$ ways to do this.
\end{itemize}
Hence there are $6 \times 3 \times 3 = 54$ options in total.
\end{example}

This may feel informal, but really what we are doing is establishing a bijection. Letting $X$ be the set of options, the above procedure defines a bijection
\[ F \times C \times S \to X \]
where $F$ is the set of flavours, $C = \{ \text{tub}, \text{regular cone}, \text{choco-cone} \}$ and $S = [3]$ is the set of possible numbers of scoops.

\begin{example}
\label{exNumSubsetsOfFiniteSet}
We will prove that $|\mathcal{P}(X)| = 2^{|X|}$ for all finite sets $X$.

Let $X$ be a finite set and let $n=|X|$. Write
\[ X = \{ x_k \mid k \in [n] \} = \{ x_0, x_1, \dots, x_{n-1} \} \]
Intuitively, specifying an element of $\mathcal{P}(X)$---that is, a subset $U \subseteq X$---is equivalent to deciding, for each $k \in [n]$, whether $x_k \in U$ or $x_k \not \in U$ (`in or out'), which in turn is equivalent to specifying an element of $\{ \text{in}, \text{out} \}^n$.

For example, taking $X=[7]$, the subset $U = \{ 1, 2, 6 \}$ corresponds with the choices
\[ 0 \text{ out},\ 1 \text{ in},\ 2 \text{ in},\ 3 \text{ out},\ 4 \text{ out},\ 5 \text{ out},\ 6 \text{ in} \]
and hence the element $(\text{out}, \text{in}, \text{in}, \text{out}, \text{out}, \text{out}, \text{in}) \in \{ \text{in}, \text{out} \}^7$.

This defines a function $i : \mathcal{P}(X) \to \{ \text{in}, \text{out} \}^n$. This function is injective, since different subsets determine different sequences; and it is surjective, since each sequence determines a subset.

The above argument is sufficient for most purposes, but since this is the first bijective proof we have come across, we now give a more formal presentation of the details.

Define a function
\[ i : \mathcal{P}(X) \to \{ \text{in},\text{out} \}^n \]
by letting the $k^{\text{th}}$ component of $i(U)$ be `in' if $x_k \in U$ or `out' if $x_k \not \in U$, for each $k \in [n]$.

We prove that $i$ is a bijection.
\begin{itemize}
\item \textbf{$i$ is injective.} To see this, take $U,V \subseteq X$ and suppose $i(U) = i(V)$. We prove that $U=V$. So fix $x \in X$ and let $k \in [n]$ be such that $x=x_k$. Then
\begin{align*}
x \in U &\Leftrightarrow \text{the $k^{\text{th}}$ component of $i(U)$ is `in'} && \text{by definition of $i$} \\
&\Leftrightarrow \text{the $k^{\text{th}}$ component of $i(V)$ is `in'} && \text{since $i(U)=i(V)$} \\
&\Leftrightarrow x \in V && \text{by definition of $i$}
\end{align*}
so indeed we have $U=V$, as required.
\item \textbf{$i$ is surjective.} To see this, let $v \in \{ \text{in},\text{out} \}^n$, and let
\[ U = \{ x_k \mid \text{the } k^{\text{th}} \text{ component of } v \text{ is `in'} \} \]
Then $i(U)=v$, since for each $k \in [n]$ we have $x_k \in U$ if and only if the $k^{\text{th}}$ component of $v$ is `in', which is precisely the definition of $i(U)$.
\end{itemize}

Hence
\[ |\mathcal{P}(X)| = |\{\text{in},\text{out}\}|^n = 2^n \]
as required.
\end{example}

Some authors actually write $2^X$ to refer to the power set of a set $X$. This is justified by \Cref{exNumSubsetsOfFiniteSet}.

\begin{exercise}
Let $X$ and $Y$ be finite sets, and recall that $Y^X$ denotes the set of functions from $X$ to $Y$. Prove that $|Y^X|=|Y|^{|X|}$.
\hintlabel{exSizeOfFunctionSet}{%
Any function $f : X \to Y$ with finite domain can be specified by listing its values. For each $x \in X$, how many choices do you have for the value $f(x)$?
}
\end{exercise}

\begin{exercise}
Since September 2001, car number plates on the island of Great Britain have taken the form \texttt{XX\;NN\;XXX}, where each \texttt{X} can be any letter of the alphabet except for `I', `Q' or `Z', and \texttt{NN} is the last two digits of the year. [This is a slight simplification of what is really the case, but let's not concern ourselves with \textit{too} many details!] How many possible number plates are there? Number plates of vehicles registered in the region of Yorkshire begin with the letter `Y'. How many Yorkshire number plates can be issued in a given year?
\end{exercise}

The multiplication principle in the form of \Cref{thmMultiplicationPrincipleIndependent} does not allow for steps later in a procedure to depend on those earlier in the procedure. To see why this is a problem, suppose we want to count the size of the set $X = \{ (a,b) \in [n] \times [n] \mid a \ne b \}$. A step-by-step procedure for specifying such an element is as follows:
\begin{itemize}
\item \textbf{Step 1.} Select an element $a \in [n]$. There are $n$ choices.
\item \textbf{Step 2.} Select an element $b \in [n]$ with $b \ne a$. There are $n-1$ choices.
\end{itemize}
We would like to use \Cref{thmMultiplicationPrincipleIndependent} to deduce that $|X| = n(n-1)$. Unfortunately, this is not valid because the possible choices available to us in Step 2 depend on the choice made in Step 1. Elements of cartesian products do not depend on one another, and so the set of sequences of choices made cannot necessarily be expressed as a cartesian product of two sets. Thus we cannot apply \Cref{lemMultiplicationPrincipleIndependent}. Oh no!

However, provided that the \textit{number} of choices in each step remains constant, in spite of the choices themselves changing, it turns out that we can still compute the size of the set in question by multiplying together the numbers of choices.

This is what we prove next. We begin with a pairwise version (analogous to \Cref{exSizeOfCartesianProduct}) and then prove the general version by induction (like in \Cref{lemMultiplicationPrincipleIndependent}).

\begin{lemma}
\label{lemMultiplicationPrinciplePairwise}
Fix $m,n \in \mathbb{N}$. Let $X$ be a finite set with $|X|=m$, and for each $a \in X$, let $Y_a$ be a finite set with $|Y_a|=n$. Then the set
\[ Z = \{ (a, b) \mid a \in X \text{ and } b \in Y_a \} \]
is finite and $|Z| = mn$.
\end{lemma}

\begin{cproof}
Fix bijections $f : [m] \to X$ and $g_a : [n] \to Y_a$ for each $a \in X$. Define $h : [m] \times [n] \to Z$ by letting $h(i,j) = (f(i), g_{f(i)}(j))$ for each $(i,j) \in [m] \times [n]$. Then:
\begin{itemize}
\item $h$ is well-defined, since for all $i \in [m]$ and $j \in [n]$ we have $f(i) \in X$ and $g_{f(i)}(j) \in Y_{f(i)}$.
\item $h$ is injective. To see this, fix $(i,j), (k,\ell) \in [m] \times [n]$ and assume that $h(i,j) = h(k,\ell)$. Then $(f(i),g_{f(i)}(j)) = (f(k),g_{f(k)}(\ell))$, so that $f(i)=f(k)$ and $g_{f(i)}(j) = g_{f(k)}(\ell)$. Since $f$ is injective, we have $i=k$---therefore $g_{f(i)}(j) = g_{f(i)}(\ell)$, and then since $g_{f(i)}$ is injective, we have $j=\ell$. Thus $(i,j) = (k,\ell)$, as required.
\item $h$ is surjective. To see this, let $(a,b) \in Z$. Since $f$ is surjective and $a \in X$, we have $a=f(i)$ for some $i \in [m]$. Since $g_a$ is surjective and $b \in Y_a$, we have $b=g_a(j)$ for some $j \in [n]$. But then
\[ (a,b) = (f(i), g_a(j)) = (f(i), g_{f(i)}(j)) = h(i,j) \]
so that $h$ is surjective.
\end{itemize}

Since $h$ is a bijection, we have $|Z| = |[m] \times [n]|$ by \Cref{thmFiniteSetsAndJections}(iii), which is equal to $mn$ by \Cref{propProductOfFiniteSetsIsFinite}.
\end{cproof}

We are now ready to state and prove the theorem giving rise to the multiplication principle in its full generality.

\begin{theorem}[Multiplication principle {(general version)}]
\label{thmMultiplicationPrincipleGeneral}
\index{counting principle!multiplication principle}
\index{multiplication principle}
\index{rule of product}
Let $X$ be a set, and suppose that there is a specification procedure for $X$ such that in each step, there are finitely many choices available, and the number of available choices is independent of those made in other steps. Then $X$ is finite.

Moreover, letting $r \in \mathbb{N}$ be the number of steps, and letting $m_k$ for $k \in [r]$ be the numbers of choices available in the respective steps, we have $|X| = \prod_{k \in [r]} m_k$.
\end{theorem}

\begin{cproof}
Given an $r$-step specification procedure for $X$, define a family of sets as follows:
\begin{itemize}
\item Let $X^{(0)}$ be the set of all choices available in Step $0$, and let $m_0 = |X^{(0)}| \in \mathbb{N}$.
\item For each $k \in \mathbb{N}$ with $0 < k < r$, and each sequence $s$ of choices available in Steps $0$ through $k-1$: let $X^{(k)}_s$ be the set of all choices available in Step $k$ when the choices made in the previous steps were those in $s$; and let $m_k = |X^{(k)}_s| \in \mathbb{N}$. Note that the value of $m_k$ is independent of $s$ by the assumption in the theorem.
\end{itemize}
Finally, let $P$ be the set of all sequences of steps. Note that there is a bijection $P \to X$ by definition of specification procedure, so it suffices to prove that $P$ is finite and that $|P| = \displaystyle \prod_{k \in [r]} m_k$.

We proceed by induction on $r \ge 1$.
\begin{itemize}
\item (\textbf{Base case}) When $r=1$, there is a single step in the specification procedure, so the function $f : X^{(0)} \to P$ defined by $f(s) = (s)$ is a bijection, and so $P$ is finite and
\[ |P| = |X^{(0)}| = m_0 = \prod_{k \in [1]} m_k \]
as required.

\item (\textbf{Induction step}) Let $r \ge 1$ and suppose that the theorem statement holds for $r$-step specification procedures.

Suppose now that we have an $(r+1)$-step specification procedure (with steps numbered $0$ through $r$, say), and let the sets $X^{(k)}_s$ and natural numbers $m_k$ (for suitable natural numbers $k$ and sequences $s$ of choices) and $P$ be defined as at the beginning of the theorem.

Let $Q$ be the set of all sequences of choices made in Steps $0$ through $r-1$, and define $Z = \{ (s,c_r) \mid s \in Q \text{ and } c_r \in X^{(r)}_s \}$.

There is an evident bijection $f : Z \to P$ given by
\[ f((c_0,\dots,c_{r-1}),c_r) = (c_0,c_1,\dots,c_{r-1},c_r) \]
for all $((c_0,\dots,c_{r-1}),c_r) \in Z$.

But now $|Q| = \displaystyle \prod_{k \in [r]} m_k$ by the induction hypothesis, and $|X^{(r)}_s| = m_r$ for all $s \in Q$ by definition of $m_r$, so by \Cref{lemMultiplicationPrinciplePairwise} we have that $Z$ is finite and
\[ |Z| = |Q| \cdot |X^{(r)}_s| = \left(\prod_{k \in [r]} m_k\right) \cdot m_r = \prod_{k \in [r+1]} m_k \]

Since $f$ is a bijection, $P$ is also finite and also has this size, as required. \qed
\end{itemize}
\end{cproof}

\begin{example}
We prove that there are six bijections $[3] \to [3]$. We can specify a bijection $f : [3] \to [3]$ according to the following procedure.
\begin{itemize}
\item \textbf{Step 1.} Choose the value of $f(0)$. There are $3$ choices: $0$, $1$ or $2$.
\item \textbf{Step 2.} Choose the value of $f(1)$. The values $f(1)$ can take depend on the chosen value of $f(0)$.
\begin{itemize}
\item If $f(0)=0$, then $f(1)$ can be equal to $1$ or $2$.
\item If $f(0)=1$, then $f(1)$ can be equal to $0$ or $2$.
\item If $f(0)=2$, then $f(1)$ can be equal to $0$ or $1$.
\end{itemize}
In each case, there are $2$ choices for the value of $f(1)$.
\item \textbf{Step 3.} Choose the value of $f(2)$. The values $f(2)$ can take depend on the values of $f(0)$ and $f(1)$. We could split into the (six!) cases based on the values of $f(0)$ and $f(1)$ chosen in Steps 1 and 2; but we won't. Instead, note that $\{f(0),f(1)\}$ has two elements, and by injectivity $f(2)$ must have a distinct value, so that $[3] \setminus \{ f(0),f(1) \}$ has one element. This element must be the value of $f(2)$. Hence there is only possible choice of $f(2)$.
\end{itemize}
By the multiplication principle, there are $3 \times 2 \times 1 = 6$ bijections $[3] \to [3]$.
\end{example}

\begin{exercise}
Count the number of injections $[3] \to [4]$.
\hintlabel{exNumberOfInjectionsThreeToFour}{%
The image (\Cref{defImage}) of an injection $[3] \to [4]$ must be a subset of $[4]$ of size three.
}
\end{exercise}

\begin{example}
We count the number of ways we can shuffle a standard deck of cards in such a way that the colour of the cards alternate between red and black.

A procedure for choosing the order of the cards is as follows:
\begin{enumerate}[(i)]
\item Choose the colour of the first card. There are $2$ such choices. This then determines the colours of the remaining cards, since they have to alternate.
\item Choose the order of the red cards. There are $26!$ such choices.
\item Choose the order of the black cards. There are $26!$ such choices.
\end{enumerate}
By the multiplication principle, there are $2 \cdot (26!)^2$ such rearrangements. This number is huge, and in general there is no reason to write it out. Just for fun, though:
\[325\;288\;005\;235\;264\;929\;014\;077\;766\;819\;257\;214\;042\;112\;000\;000\;000\;000\]
\end{example}

\subsection*{Sums and partitions}

We saw in \Cref{propUnionOfFiniteSetsIsFinite} that, given two finite sets $X$ and $Y$, the union $X \cup Y$ is finite. We also found formulae for their size, namely $|X \cup Y| = |X| + |Y| - |X \cap Y|$. The \textit{addition principle} (\Cref{thmAdditionPrinciple}) generalises this formula to any finite number of sets, provided the sets have no elements in common with one another---that is they are \textit{pairwise disjoint}. [The hypothesis of pairwise disjointness is removed in the \textit{inclusion--exclusion principle}, which is studied in \Cref{secAlternatingSums}.]

If you have not covered \Cref{secEquivalenceRelationsPartitions} yet, you are encouraged to take a brief detour to read from \Cref{defPartition} to \Cref{exConditionsForPartition}; the definition of a \textit{partition} of a set is recalled below.

\rdefPartition*

In this section, we will simplify matters in two ways:
\begin{itemize}
\item When we say `partition' in this section (and \Cref{secAlternatingSums}), we will allow the sets in the partition to be empty---that is, we will just need conditions (b) and (c) of \Cref{defPartition} to hold.
\item Since our sets are finite, so will $\mathcal{A}$ be; so we will only ever partition our sets into finitely many pieces. That is, all of our partitions will take form $\mathcal{A} = \{ A_i \mid i \in [n] \}$ for some $n \in \mathbb{N}$.
\end{itemize}

With all of this said, let's get right to it.

\begin{theorem}[Addition principle]
\label{thmAdditionPrinciple}
\index{addition principle}
\index{rule of sum}
\index{counting principle!addition principle}
Let $X$ be a set and let $\mathcal{A} = \{ A_i \mid i \in [n] \}$ be a partition of $X$ for some $n \in \mathbb{N}$, such that each set $A_i$ is finite. Then $X$ is finite, and
\[ |X| = \sum_{i \in [n]} |A_i| \]
\end{theorem}

\begin{exercise}
Prove \Cref{thmAdditionPrinciple}. The proof follows the same pattern as that of \Cref{lemMultiplicationPrincipleIndependent}. Be careful to make sure you identify where you use the hypothesis that the sets $U_i$ are pairwise disjoint!
\end{exercise}

\begin{example}
We will count the number of non-empty subsets of $[7]$ which either contain only even numbers, or contain only odd numbers.

Let $E$ denote the set of non-empty subsets of $[7]$ containing only even numbers, and let $O$ denote the set of non-empty subsets of $[7]$ containing only odd numbers. Note that $\{ E, O \}$ forms a partition of the set we are counting, and so our set has $|E|+|O|$ elements.
\begin{itemize}
\item An element of $E$ must be a subset of $\{0,2,4,6\}$. By \Cref{exNumSubsetsOfFiniteSet} there are $2^4=16$ such subsets. Thus the number of \textit{non-empty} subsets of $[7]$ containing only odd numbers is $15$, since we must exclude the empty set. That is, $|E|=15$.
\item An element of $O$ must be a subset of $\{1,3,5\}$. Again by \Cref{exNumSubsetsOfFiniteSet} there are $2^3=8$ such subsets. Hence there are $7$ non-empty subsets of $[7]$ containing only even numbers. That is, $|O|=7$.
\end{itemize}
Hence there are $15+7=22$ non-empty subsets of $[7]$ containing only even or only odd numbers. And here they are:
\[ \begin{matrix}
\{ 0 \} & \{ 2 \} & \{ 4 \} & \{ 6 \} & \{ 0, 2 \} & \hspace{20pt} & \{ 1 \} & \{ 3 \} & \{ 5 \} \\
\{ 0, 4 \} & \{ 0, 6 \} & \{ 2, 4 \} & \{ 2, 6 \} & \{ 4, 6 \} && \{ 1, 3 \} & \{ 1, 5 \} & \{ 3, 5 \} \\
\{ 0, 2, 4 \} & \{ 0, 2, 6 \} & \{ 0, 4, 6 \} & \{ 2, 4, 6 \} & \{ 0, 2, 4, 6\} && \{ 1, 3, 5 \} && 
\end{matrix} \]
\end{example}

\begin{exercise}
Pick your favourite integer $n \ge 1000$. For this value of $n$, how many non-empty subsets of $[n]$ contain either only even or only odd numbers? (You need not evaluate exponents.)
\end{exercise}

We now consider some examples of finite sets which use both the multiplication principle and the addition principle.

\begin{example}
\label{exCityColour}
A city has $6n$ inhabitants. The favourite colour of $n$ of the inhabitants is orange, the favourite colour of $2n$ of the inhabitants is pink, and the favourite colour of $3n$ of the inhabitants is turquoise. The city government wishes to form a committee with equal representation from the three colour preference groups to decide how the new city hall should be painted. We count the number of ways this can be done.

Let $X$ be the set of possible committees. First note that
\[ X = \bigcup_{k=0}^n A_k \]
where $A_k$ is the set of committees with exactly $k$ people from each colour preference group. Indeed, we must have $k \le n$, since it is impossible to have a committee with more than $n$ people from the orange preference group.

Moreover, if $k \ne \ell$ then $A_k \cap A_{\ell} = \varnothing$, since if $k \ne \ell$ then a committee cannot simultaneously have exactly $k$ people and exactly $\ell$ people from each preference group.

By the addition principle, we have
\[ |X| = \sum_{k=0}^n |A_k| \]
We count $A_k$ for fixed $k$ using the following procedure:
\begin{itemize}
\item \textbf{Step 1.} Choose $k$ people from the orange preference group to be on the committee. There are $\binom{n}{k}$ choices.
\item \textbf{Step 2.} Choose $k$ people from the pink preference group to be on the committee. There are $\binom{2n}{k}$ choices.
\item \textbf{Step 3.} Choose $k$ people from the turquoise preference group to be on the committee. There are $\binom{3n}{k}$ choices.
\end{itemize}
By the multiplication principle, it follows that $|A_k| = \binom{n}{k} \binom{2n}{k} \binom{3n}{k}$. Hence
\[ |X| = \sum_{k=0}^n \binom{n}{k} \binom{2n}{k} \binom{3n}{k} \]
\end{example}

\begin{exercise}
\label{exCityColourModified}
In \Cref{exCityColour}, how many ways could a committee be formed with a \textit{representative} number of people from each colour preference group? That is, the proportion of people on the committee which prefer any of the three colours should be equal to the corresponding proportion of the population of the city.
\end{exercise}

\subsection*{Pigeonhole principle}

A nice application of the addition principle is to prove the \textit{pigeonhole principle}, which is used heavily in combinatorics.

Informally, the pigeonhole principle says that if you assign pigeons to pigeonholes, and there are more pigeons than pigeonholes, then some pigeonhole must have more than one pigeon in it. We can (and do) generalise this slightly: it says that given $q \in \mathbb{N}$, if you have more than $q$ times as many pigeons than pigeonholes, then some pigeonhole must have more than $q$ pigeons in it.

The proof is deceptively simple.

\begin{theorem}[Pigeonhole principle]
\label{thmPigeonholePrinciple}
\index{pigeonhole principle}
Let $q \in \mathbb{N}$, and let $X$ and $Y$ be finite sets with $|X| = m \in \mathbb{N}$ and $|Y| = n \in \mathbb{N}$. Then:
\begin{enumerate}[(a)]
\item If $m > qn$, then for every function $f : X \to Y$, there is some $a \in Y$ such that $|f^{-1}[\{a\}]| > q$.
\item If $m \le qn$, then there is a function $f : X \to Y$ such that $|f^{-1}[\{a\}]| \le q$ for all $a \in Y$.
\end{enumerate}
\end{theorem}

\begin{cproof}[of {(a)}]
Suppose $m>qn$. It follows from \Cref{exPreimagesFormPartition} that the sets $f^{-1}[\{a\}]$ partition $X$. Towards a contradiction, assume $|f^{-1}[\{a\}]| \le q$ for all $a \in Y$. Then by the addition principle
\[ m ~=~ |X| ~=~ \left| \bigcup_{a \in Y} f^{-1}[\{a\}] \right| ~=~ \sum_{a \in Y} |f^{-1}[\{a\}]| ~\le~ \sum_{a \in Y} q ~=~ |Y| \cdot q ~=~ qn \]
This contradicts the assumption that $m > qn$.
\end{cproof}

\begin{exercise}
Prove part (b) of \Cref{thmPigeonholePrinciple}.
\end{exercise}

\begin{example}
Let $n, k \in \mathbb{N}$. Assume that you have $n$ pairs of socks in a drawer, and each sock is one of $k$ colours. We wish to know how many socks you must take out of the drawer before you can guarantee that you have a matching pair.

Let $C$ be set of colours of the socks, so that $|C| = k$, and let $X$ be the set of socks that you have selected. We obtain a function $f : X \to C$ that assigns to each sock $x$ its colour $f(x) \in C$. Given a colour $c \in C$, the preimage $f^{-1}[\{c\}]$ is the set of socks of colour $c$ that we have selected.

Thus the question becomes: what size must $X$ be in order to have $|f^{-1}[\{c\}]| \ge 2$ for some $c \in C$? [The English translation of this question is: how many socks must we have picked in order for two of the socks to have the same colour?]

Well, by the pigeonhole principle, we can guarantee $|f^{-1}[\{c\}]| \ge 2$ (or equivalently $>1$) if and only if $|X| > |C|$. That is, we need to select at least $k + 1$ socks to guarantee a matching pair.
\end{example}

\begin{exercise}
Six people are in a room. The atmosphere is tense, since each pair of people is either friends or enemies. There are no allegiances, so for example it is possible for a friend of a friend to be an enemy, or an enemy of a friend to be a friend, and so on. Prove that there is some set of three people that are either all each other's friends or all each other's enemies.
\end{exercise}

\subsection*{Double counting}
% \subsection*{Counting in two ways}

\textit{Double counting} (also known as \textit{counting in two ways})
% \textit{Counting in two ways} (also known as \textit{double counting})
is a proof technique that allows us to prove that two natural numbers are equal by establishing they are two expressions for the size of the same set. (More generally, by \Cref{thmFiniteSetsAndJections}(iii), we can relate them to the sizes of two sets which are in bijection.)

The proof of \Cref{propSumOfBinomialsIsExp} illustrates this proof very nicely. We proved it already by induction in \Cref{exSumOfBinomialCoefficients}; the combinatorial proof we now provide is much shorter and cleaner.

\begin{proposition} \label{propSumOfBinomialsIsExp}
Let $n \in \mathbb{N}$. Then $2^n = \displaystyle\sum_{k=0}^n \binom{n}{k}$.
\end{proposition}
\begin{cproof}
We know that $|\mathcal{P}([n])| = 2^n$ by \Cref{exNumSubsetsOfFiniteSet} and that $\mathcal{P}([n]) = \bigcup_{k=0}^n \binom{[n]}{k}$ by \Cref{propUnionOfFinSubsetsEqPowerSet}. Moreover, the sets $\binom{[n]}{k}$ are pairwise disjoint, so by the addition principle it follows that
\[ 2^n = |\mathcal{P}([n])| = \left|\bigcup_{k=0}^n \binom{[n]}{k}\right| = \sum_{k=0}^n \left| \binom{[n]}{k} \right| = \sum_{k=0}^n \binom{n}{k} \]
\end{cproof}

\begin{strategy}[Double counting]
% \begin{strategy}[Counting in two ways]
\index{double counting}
% \index{counting in two ways}
In order to prove that two expressions involving natural numbers are equal, it suffices to define a set $X$ and devise two counting arguments to show that $|X|$ is equal to both expressions.
\end{strategy}

The next example counts elements of \textit{different} sets and puts them in bijection to establish an identity.

\begin{proposition}
\label{propBinomialSymmetry}
Let $n,k \in \mathbb{N}$ with $n \ge k$. Then \[ \binom{n}{k} = \binom{n}{n-k} \]
\end{proposition}
\begin{cproof}
%% BEGIN EXTRACT (xtrAssumingImplicationsExampleTwo) %%
First note that $\binom{n}{k} = \left| \binom{[n]}{k} \right|$ and $\binom{n}{n-k} = \left| \binom{[n]}{n-k} \right|$, so in order to prove $\binom{n}{k} = \binom{n}{n-k}$, it suffices by \Cref{strComparingSizesOfFiniteSets} to find a bijection $f : \binom{[n]}{k} \to \binom{[n]}{n-k}$.
%% END EXTRACT %%
Intuitively, this bijection arises because choosing $k$ elements from $[n]$ to \textit{put into} a subset is equivalent to choosing $n-k$ elements from $[n]$ to \textit{leave out of} the subset. Specifically, we define
\[ f(U) = [n] \setminus U \text{ for all } U \in \binom{[n]}{k} \]
Note first that $f$ is well-defined, since if $U \subseteq [n]$ with $|U|=k$, then $[n] \setminus U \subseteq [n]$ and $|[n] \setminus U| = |[n]|-|U| = n-k$ by \Cref{exSizeOfRelativeComplement}. We now prove $f$ is a bijection:
\begin{itemize}
\item \textbf{$f$ is injective.} Let $U, V \subseteq [n]$ and suppose $[n] \setminus U = [n] \setminus V$. Then for all $k \in [n]$, we have
\begin{align*}
k \in U &\Leftrightarrow k \not \in [n] \setminus U && \text{by definition of set difference} \\
&\Leftrightarrow k \not \in [n] \setminus V && \text{since $[n] \setminus U = [n] \setminus V$} \\
&\Leftrightarrow k \in V && \text{by definition of set difference}
\end{align*}
so $U=V$, as required.
\item \textbf{$f$ is surjective.} Let $V \in \binom{[n]}{n-k}$. Then $|[n] \setminus V| = n-(n-k) = k$ by \Cref{exSizeOfRelativeComplement}, so that $[n] \setminus V \in \binom{[n]}{k}$. But then
\[ f([n] \setminus V) = [n] \setminus ([n] \setminus V) = V \]
by \Cref{exSetMinusSetMinus}.
\end{itemize}
Since $f$ is a bijection, we have
\[ \binom{n}{k} = \left| \binom{[n]}{k} \right| = \left| \binom{[n]}{n-k} \right| = \binom{n}{n-k} \]
as required.
\end{cproof}

We put a lot of detail into this proof. A slightly less formal proof might simply say that $\binom{n}{k} = \binom{n}{n-k}$ since choosing $k$ elements from $[n]$ to put into a subset is equivalent to choosing $n-k$ elements from $[n]$ to leave out of the subset. This would be fine as long as the members of the intended audience of your proof could reasonably be expected to construct the bijection by themselves.

The proof of \Cref{propBinomCoeffTwoColourBalls} follows this more informal format.

\begin{proposition} \label{propBinomCoeffTwoColourBalls}
Let $n,k,\ell \in \mathbb{N}$ with $n \ge k \ge \ell$. Then
\[ \binom{n}{k}\binom{k}{\ell} = \binom{n}{\ell}\binom{n-\ell}{k-\ell} \]
\end{proposition}

\begin{cproof}
Let's home in on the left-hand side of the equation. By the multiplication principle, $\binom{n}{k} \binom{k}{\ell}$ is the number of ways of selecting a $k$-element subset of $[n]$ and an $\ell$-element subset of $[k]$. Equivalently, it's the number of ways of selecting a $k$-element subset of $[n]$ and then an $\ell$-element subset \textit{of the $k$-element subset that we just selected}. To make this slightly more concrete, let's put it this way:
\begin{quote}
$\binom{n}{k} \binom{k}{\ell}$ is the number of ways of painting $k$ balls red from a bag of $n$ balls, and painting $\ell$ of the red balls blue. This leaves us with $\ell$ blue balls and $k-\ell$ red balls.
\end{quote}
Now we need to find an equivalent interpretation of $\binom{n}{\ell} \binom{n-\ell}{k-\ell}$. Well, suppose we pick the $\ell$ elements to be coloured blue first. To make up the rest of the $k$-element subset, we now have to select $k-\ell$ elements, and there are now $n-\ell$ to choose from. Thus
\begin{quote}
$\binom{n}{\ell} \binom{n-\ell}{k-\ell}$ is the number of ways of painting $\ell$ balls from a bag of $n$ balls blue, and painting $k-\ell$ of the remaining balls red.
\end{quote}
Thus, both numbers represent the number of ways of painting $\ell$ balls blue and $k-\ell$ balls red from a bag of $n$ balls. Hence they are equal.
\end{cproof}

\begin{exercise}
Make the proof of \Cref{propBinomCoeffTwoColourBalls} more formal by defining a bijection between sets of the appropriate sizes.
\end{exercise}

\begin{exercise}
\label{exPascalIdentity}
Provide a combinatorial proof that if $n,k \in \mathbb{N}$ with $n \ge k$, then
\[ \binom{n+1}{k+1} = \binom{n}{k} + \binom{n}{k+1} \]

Deduce that the combinatorial definition of binomial coefficients (\Cref{defBinomialCoefficient}) is equivalent to the recursive definition (\Cref{defBinomialCoefficientRecursive}).
\begin{backhint}
\hintref{exPascalIdentity}
How many ways can you select $k+1$ animals from a set containing $n$ cats and one dog?
\end{backhint}
\end{exercise}

The following proposition demonstrates that the combinatorial definition of factorials (\Cref{defFactorial}) is equivalent to the recursive definition (\Cref{defFactorialRecursive}).

\begin{theorem}
\label{thmFactorialAsProduct}
$0!=1$, and $(n+1)! = (n+1) \cdot n!$ for all $n \in \mathbb{N}$.
\end{theorem}
\begin{cproof}
The only permutation of $\varnothing$ is the empty function $e : \varnothing \to \varnothing$. Hence $S_0 = \{ e \}$ and $0!=|S_0|=1$.

Let $n \in \mathbb{N}$. A permutation of $[n+1]$ is a bijection $f : [n+1] \to [n+1]$. Specifying such a bijection is equivalent to carrying out the following procedure:
\begin{itemize}
\item Choose the (unique!) element $k \in [n+1]$ such that $f(k) = n$. There are $n+1$ choices for $k$.
\item Choose the values of $f$ at each $\ell \in [n+1]$ with $\ell \ne k$. This is equivalent to finding a bijection $[n+1] \setminus \{ k \} \to [n]$. Since $|[n+1] \setminus \{ k \}| = |[n]| = n$, there are $n!$ such choices.
\end{itemize}
By the multiplication principle, we have
\[ (n+1)! = |S_{n+1}| = (n+1) \cdot n! \]
so we're done.
\end{cproof}

We now revisit \Cref{thmBinomAsFactorialByInduction}; this time, our proof will be combinatorial, rather than inductive.

\begin{theorem} \label{thmBinomAsFactorial}
Let $n, k \in \mathbb{N}$. Then
\[ \binom{n}{k} = \begin{cases} \dfrac{n!}{k!(n-k)!} & \text{if } k \le n \\ 0 & \text{if } k > n \end{cases} \]
\end{theorem}
\begin{cproof}
Suppose $k>n$. By \Cref{exSubsetOfFiniteSetIsFinite}, if $U \subseteq [n]$ then $|U| \le n$. Hence if $k > n$, then $\binom{[n]}{k} = \varnothing$, and so $\binom{n}{k}=0$, as required.

Now suppose $k \le n$. We will prove that $n! = \binom{n}{k} \cdot k! \cdot (n-k)!$; the result then follows by dividing through by $k!(n-k)!$. We prove this equation by counting the number of elements of $S_n$.

A procedure for defining an element of $S_n$ is as follows:
\begin{enumerate}[(i)]
\item Choose which elements will appear in the first $k$ positions of the list. There are $\binom{n}{k}$ such choices.
\item Choose the order of these $k$ elements. There are $k!$ such choices.
\item Choose the order of the remaining $n-k$ elements. There are $(n-k)!$ such choices.
\end{enumerate}
By the multiplication principle, $n! = \binom{n}{k} \cdot k! \cdot (n-k)!$.
\end{cproof}

Note that the proof of \Cref{thmBinomAsFactorial} relied only on the combinatorial definitions of binomial coefficients and factorials; we didn't need to know how to compute them at all! The proof was \textit{much} shorter, cleaner and, in some sense, more meaningful, than the inductive proof we gave in \Cref{thmBinomAsFactorialByInduction}.

We conclude this section with some more examples and exercises in which
double counting
% counting in two ways
can be used.

\begin{exercise}
\label{exCountingKTimesNChooseK}
Let $n,k \in \mathbb{N}$ with $k \le n+1$. Prove that
\[ k \binom{n}{k} = (n-k+1) \binom{n}{k-1} \]
\begin{backhint}
\hintref{exCountingKTimesNChooseK}
Find two procedures for counting the number of pairs $(U, u)$, such that $U \subseteq [n]$ is a $k$-element subset and $u \in U$. Equivalently, count the number of ways of forming a committee of size $k$ from a population of size $n$, and then appointing one member of the committee to be the chair.
\end{backhint}
\end{exercise}

\begin{example}
\label{exCombinatorialIdentityCatsAndDogs}
Let $m,n,k \in \mathbb{N}$. We prove that
\[ \sum_{\ell=0}^k \binom{m}{\ell} \binom{n}{k-\ell} = \binom{m+n}{k} \]
by finding a procedure for counting the number of $k$-element subsets of an appropriate $(m+n)$-element set. Specifically, let $X$ be a set containing $m$ cats and $n$ dogs. Then $\left| \binom{m+n}{k} \right|$ is the number of $k$-element subsets $U \subseteq X$. We can specify such a subset according to the following procedure.
\begin{itemize}
\item \textbf{Step 1.} Split into cases based on the number $\ell$ of cats in $U$. Note that we must have $0 \le \ell \le k$, since the number of cats must be a natural number and cannot exceed $k$ as $|U|=k$. Moreover, these cases are mutually exclusive. Hence by the addition principle we have
\[ \binom{m+n}{k} = \sum_{\ell=0}^k a_{\ell} \]
where $a_{\ell}$ is the number of subsets of $X$ containing $\ell$ cats and $k-\ell$ dogs.
\item \textbf{Step 2.} Choose $\ell$ cats from the $m$ cats in $X$ to be elements of $U$. There are $\binom{m}{\ell}$ such choices.
\item \textbf{Step 3.} Choose $k-\ell$ dogs from the $n$ dogs in $X$ to be elements of $U$. There are $\binom{n}{k-\ell}$ such choices.
\end{itemize}
The multiplication principle shows that $a_{\ell} = \binom{m}{\ell} \binom{n}{k-\ell}$. Hence
\[ \binom{m+n}{k} = \sum_{\ell=0}^k \binom{m}{\ell} \binom{n}{k-\ell} \]
as required.
\end{example}

\begin{exercise}
\label{exTrinomialCoefficients}
Given natural numbers $n,a,b,c$ with $a+b+c=n$, define the \textbf{trinomial coefficient} $\displaystyle \binom{n}{a,b,c}$\nindex{nChoosek3}{$\binom{n}{a,b,c}$}{trinomial coefficient} \index{trinomial coefficient} to be the number of ways of partitioning $[n]$ into three sets of sizes $a$, $b$ and $c$, respectively. That is, $\displaystyle \binom{n}{a,b,c}$ is the size of the set
\[ \left\{ (A,B,C)\ \middle|\ 
\begin{matrix} \begin{matrix} A \subseteq [n], & B \subseteq [n], & C \subseteq[n], \\ |A|=a, & |B|=b, & |C|=c, \end{matrix} \\ \text{and } A \cup B \cup C = [n] \end{matrix} \right\} \]
By considering trinomial coefficients, prove that if $a,b,c \in \mathbb{N}$, then $(a+b+c)!$ is divisible by $a! \cdot b! \cdot c!$.
\begin{backhint}
\hintref{exTrinomialCoefficients}
Find an expression for $(a+b+c)!$ in terms of $a!$, $b!$, $c!$ and $\binom{a+b+c}{a,b,c}$, following the pattern of \Cref{thmBinomAsFactorial}.
\end{backhint}
\end{exercise}

\index{counting|)}