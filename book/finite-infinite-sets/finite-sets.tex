% !TeX root = ../../infdesc.tex
\section{Finite sets}
\secbegin{secFiniteSets}

As its title suggests, this section is all about exploring the properties of finite sets, and to do this we must first define what we mean by `finite'. We certainly know a finite set when we see one---for example:
\begin{itemize}
\item The set $\{ \text{red}, \text{orange}, \text{yellow}, \text{green}, \text{blue}, \text{purple} \}$ is finite.
\item The set $[0,1]$ is infinite, but it has finite length.
\item The set $[0,\infty)$ is infinite and has infinite length.
\item The set $\mathcal{P}(\mathbb{N})$ is infinite, but has no notion of `length' to speak of.
\item The empty set $\varnothing$ is finite.
\end{itemize}

If we are to make a definition of `finite set', we must first figure out what the finite sets above have in common but the infinite sets do not.

It is difficult to define `finite' without being imprecise. A first attempt at a definition might be something like the following:
\begin{center}
\textit{A set $X$ is finite if the elements of $X$ don't go on forever.}
\end{center}
This is good intuition, but isn't good enough as a mathematical definition, because `go on' and `forever' are not precise terms (unless they themselves are defined). So let's try to make this more precise:
\begin{center}
\textit{A set $X$ is finite if the elements of $X$ can be listed one by one\\
in such a way that the list has both a start and an end.}
\end{center}
This is better but is still not entirely precise---it is not entirely clear what is meant by `listed one by one'. But we can make this precise. Let's label the positions in our list with natural numbers. To list the elements of $X$ means to specify an element of $X$ to go in position $0$, then another element of $X$ to go in position $1$, and so on, until we eventually run out of elements of $X$---the fact that we eventually run out of elements is exactly what we want to capture by our definition of `finite'.

If the listed elements of $X$ occupy positions $0,1,\dots,{n-1}$, this means that we have effectively paired up the elements of $[n]$ (the list positions) with the elements of $X$ (the set being listed). Since `pairing up' really means `finding a bijection', we are now ready to define what it means for a set to be finite.

\begin{definition}
\label{defFiniteSet}
\index{finite set}
\index{infinite set}
\index{enumeration!of a finite set}
A set $X$ is \textbf{finite} if there exists a bijection $f : [n] \to X$ for some $n \in \mathbb{N}$. The function $f$ is called an \textbf{enumeration} of $X$. If $X$ is not finite we say it is \textbf{infinite}.
\end{definition}

This definition suggests the following strategy for proving that a set is finite.

\begin{strategy}[Proving that a set is finite]
In order to prove that a set $X$ is finite, it suffices to find a bijection $[n] \to X$ for some $n \in \mathbb{N}$.
\end{strategy}

\begin{example}
\label{exSomeFiniteSets}
Let $X = \{ \text{red}, \text{orange}, \text{yellow}, \text{green}, \text{blue}, \text{purple} \}$, and define $f : [6] \to X$ by
\[ \begin{matrix}
f(0) = \text{red} & f(1) = \text{orange} & f(2) = \text{yellow} \\
f(3) = \text{green} & f(4) = \text{blue} & f(5) = \text{purple}
\end{matrix} \]
The function $f$ is evidently a bijection, since each element of $X$ can be expressed uniquely as $f(k)$ for some $k \in [6]$. So $X$ is finite.
\end{example}

\begin{exercise}
\label{exBracketNIsFinite}
Prove that $[n]$ is finite for each $n \in \mathbb{N}$.
\end{exercise}

Note that \Cref{exBracketNIsFinite} implies, in particular, that $\varnothing$ is finite, since $\varnothing = [0]$.

\subsection*{The size of a finite set}

Whilst it might sometimes be useful just to know \textit{that} a set is finite, it will be even more useful to know how many elements it has. This quantity is called the \textit{size} of the set. Intuitively, the size of the set should be the length of the list of its elements, but for this to be well-defined, we first need to know that the number of elements in the list is independent of the way we choose to list them.

The `list of elements' of a finite set $X$ is the bijection $[n] \to X$ given by \Cref{defFiniteSet}, and $n$ is the length of the list, this means that we need to prove that if $[m] \to X$ and $[n] \to X$ are bijections, then $m=n$. This will be \Cref{thmUniquenessOfSize}.

To be able to prove this, we must first prove some technical results involving functions that we will use in the proof.

\begin{lemma}
\label{lemRemoveElementFromDomainAndCodomain}
Let $X$ and $Y$ be sets, let $f : X \to Y$ be an injection and let $a \in X$. Then there is an injection $f^{\vee} : X \setminus \{ a \} \to Y \setminus \{ f(a) \}$ defined by $f^{\vee}(x) = f(x)$ for all $x \in X \setminus \{ a \}$. Moreover if $f$ is a bijection $X \to Y$, then $f^{\vee}$ is a bijection $X \setminus \{ a \} \to Y \setminus \{ f(a) \}$.
\end{lemma}

\begin{cproof}
Note that $f^{\vee}$ is well-defined by injectivity of $f$: indeed, if we had $f(x) = f(a)$ for some $x \in X \setminus \{ a \}$, then that would imply $x=a \in X \setminus \{ a \}$, which is a contradiction.

To see that $f^{\vee}$ is injective, let $x,y \in X \setminus \{ a \}$ and $f^{\vee}(x) = f^{\vee}(y)$. Then $f(x) = f(y)$, so that $x=y$ by injectivity of $f$.

Now suppose that $f$ is a bijection. Then $f$ has an inverse $g : Y \to X$, which is itself a bijection (and hence an injection). Repeating the above argument with $g$ yields an injection $g^{\vee} : Y \setminus \{ f(a) \} \to X \setminus \{ a \}$ defined by $g^{\vee}(y) = g(y)$ for all $y \in Y \setminus \{ f(a) \}$. But then for all $x \in X \setminus \{ a \}$ and all $y \in Y \setminus \{ f(a) \}$ we have
\[ g^{\vee}(f^{\vee}(x)) = g(f(x)) = x \quad \text{and} \quad f^{\vee}(g^{\vee}(y)) = f(g(y)) \]
so that $g^{\vee}$ is an inverse for $f^{\vee}$. Hence if $f$ is a bijection, then $f^{\vee}$ is a bijection.
\end{cproof}

A consequence of \Cref{lemRemoveElementFromDomainAndCodomain} is the next useful result, which we will use in our proof that every set has a unique `size'.

\begin{lemma}
\label{lemRemoveElementFromSet}
Let $X$ be an inhabited set. There is a bijection $X \setminus \{ a \} \to X \setminus \{ b \}$ for all $a, b \in X$.
\end{lemma}

\begin{cproof}
Define $f : X \to X$ be the function that swaps $a$ and $b$ and leaves all other elements of $X$ fixed---that is, we define
\[ f(x) = \begin{cases} b & \text{if } x = a \\ a & \text{if } x = b \\ x & \text{if } x \not\in \{ a, b \} \end{cases} \]
Note that $f(f(x)) = x$ for all $x \in X$, so $f$ is a bijection---it is its own inverse! Since $f(a) = b$, it follows from \Cref{lemRemoveElementFromDomainAndCodomain} that the function $f^{\vee} : X \setminus \{ a \} \to X \setminus \{ b \}$ defined by $f^{\vee}(x) = f(x)$ for all $x \in X \setminus \{ a \}$ is a bijection, as required.
\end{cproof}

\begin{theorem}
\label{thmJectionsAndSizeOfNaturalNumbers}
Let $m,n \in \mathbb{N}$.
\begin{enumerate}[(a)]
\item If there exists an injection $f : [m] \to [n]$, then $m \le n$.
\item If there exists a surjection $g : [m] \to [n]$, then $m \ge n$.
\item If there exists a bijection $h : [m] \to [n]$, then $m=n$.
\end{enumerate}
\end{theorem}

\begin{cproof}[of (a)]
We will prove by induction on $m$ that, for all $m,n \in \mathbb{N}$, if there exists an injection $[m] \to [n]$, then $m \le n$.

\begin{itemize}
\item (\textbf{Base case}) We need to prove that, for all $n \in \mathbb{N}$ if there exists an injection $[0] \to [n]$, then $0 \le n$. This is automatically true, since $0 \le n$ for all $n \in \mathbb{N}$.
\item (\textbf{Induction step}) Fix $m \in \mathbb{N}$ and suppose that, for all $n \in \mathbb{N}$, if there exists an injection $[m] \to [n]$, then $m \le n$.

Now let $n \in \mathbb{N}$ and suppose that there is an injection $f : [m+1] \to [n]$. We need to prove that $m+1 \le n$.

First note that $n > 0$. Indeed, since $m+1 > 0$, we have $0 \in [m+1]$, and so $f(0) \in [n]$. This means that $[n]$ is inhabited, and so $n > 0$. In particular, $n-1 \in \mathbb{N}$ and so the set $[n-1]$ is well-defined. It suffices to prove that $m \le n-1$.

Let $a = f(m) \in [n]$. Then:
\begin{itemize}
\item By \Cref{lemRemoveElementFromDomainAndCodomain} there is an injection $f^{\vee} : [m] = [m+1] \setminus \{ m \} \to [n] \setminus \{ a \}$.
\item By \Cref{lemRemoveElementFromSet} there is a bijection $g : [n] \setminus \{ a \} \to [n] \setminus \{ n-1 \} = [n-1]$.
\end{itemize}

Composing these two functions gives an injection $g \circ f^{\vee} : [m] \to [n-1]$. But then $m \le n-1$ by the induction hypothesis, and so $m+1 \le n$ as required.
\end{itemize}
The result now follows by induction.
\end{cproof}

\begin{exercise}
Prove parts (b) and (c) of \Cref{thmJectionsAndSizeOfNaturalNumbers}.
\hintlabel{exFinishProofOfJectionsAndSizeOfNaturalNumbers}{%
Part (b) has a proof by induction that looks much like the one in part (a). In the induction step, given a surjection $g : [m+1] \to [n]$, observe that we must have $n \ge 1$, and construct a surjection $g^- : [m+1] \setminus \{ a \} \to [n-1]$ for some suitable $a \in [m+1]$. Then invoke \Cref{lemRemoveElementFromSet}, now using the fact that $[m] = [m+1] \setminus \{m+1\}$.
}
\end{exercise}

Phew! That was fun. With these technical results proved, we can now prove the theorem we needed for the size of a finite set to be well-defined.

\begin{theorem}[Uniqueness of size]
\label{thmUniquenessOfSize}
Let $X$ be a finite set and let $f : [m] \to X$ and $g : [n] \to X$ be enumerations of $X$, where $m,n \in \mathbb{N}$. Then $m=n$.
\end{theorem}

\begin{cproof}
Since $f : [m] \to X$ and $g : [n] \to X$ are bijections, the function $g^{-1} \circ f : [m] \to [n]$ is a bijection by \Cref{exCompositeOfBijectionsIsBijection,exInverseBijection}. Hence $m=n$ by \Cref{thmJectionsAndSizeOfNaturalNumbers}(c).
\end{cproof}

As we mentioned above, \Cref{thmUniquenessOfSize} tells us that any two ways of listing (enumerating) the elements of a finite set yield the same number of elements. We may now make the following definition.

\begin{definition}
\label{defSize}
\index{size}
\index{cardinality!of a finite set}
Let $X$ be a finite set. The \textbf{size} (or \textbf{cardinality}) of $X$, written $|X|$, is the unique natural number $n$ for which there exists a bijection $[n] \to X$.
\end{definition}

\begin{example}
\Cref{exSomeFiniteSets} showed that $|\{ \text{red}, \text{orange}, \text{yellow}, \text{green}, \text{blue}, \text{purple} \}| = 6$, and provided the proof was correct, \Cref{exBracketNIsFinite} showed that $|[n]| = n$ for all $n \in \mathbb{N}$; in particular, $|\varnothing| = 0$.
\end{example}

\begin{example}
\label{exBijectionFromIntegersFromMinusNToNToBracketTwoNPlusOne}
Fix $n \in \mathbb{N}$ and let $X = \{ a \in \mathbb{Z} \mid -n \le a \le n \}$. There is a bijection $f : [2n+1] \to X$ defined by $f(k) = k-n$ for all $k \in [2n+1]$. Indeed:
\begin{itemize}
\item \textbf{$f$ is well-defined.}
%% BEGIN EXTRACT (xtrIntroducingVariableExampleTwo) %%
We need to prove $f(k) \in X$ for all $k \in [2n+1]$. Well given $k \in [2n+1]$, we have $0 \le k \le 2n$, and so
\[ -n = 0-n ~ \le ~ \underbrace{k-n}_{=f(k)} ~ \le ~ 2n - n = n \]
so that $f(k) \in X$ as claimed.
%% END EXTRACT %%
\item \textbf{$f$ is injective.} Let $k, \ell \in [2n+1]$ and assume $f(k) = f(\ell)$. Then $k-n = \ell-n$, and so $k = \ell$.
\item \textbf{$f$ is surjective.} Let $a \in X$ and define $k = a+n$. Then
\[ 0 = (-n) + n \le ~ \underbrace{a + n}_{= k} ~ \le n + n = 2n < 2n+1 \]
and so $k \in [2n+1]$, and moreover $f(k) = (a+n)-n = a$.
\end{itemize}
Since $f$ is a bijection, we have $|X| = 2n+1$ by \Cref{defSize}.
\end{example}

\begin{exercise}
\label{exAddRemoveElementsOfFiniteSets}
Let $X$ be a finite set.
\begin{enumerate}[(a)]
\item Prove that for all $a \in X$, the set $X \setminus \{ a \}$ is finite and $|X \setminus \{ a \}| = |X|-1$; and
\item Prove that for all $b \not\in X$, the set $X \cup \{ b \}$ is finite and $|X \cup \{ b \}| = |X|+1$.
\end{enumerate}
What happens in part (a) if we do not require $a \in X$? What happens in part (b) if we do not require $b \not\in X$?
\end{exercise}

\subsection*{Comparing the sizes of finite sets}

When we used dots and stars to motivate the definitions of injective and surjective functions at the beginning of \Cref{secInjectionsSurjections}, we suggested the following intuition:
\begin{itemize} 
\item If there is an injection $f : X \to Y$, then $X$ has `at most as many elements as $Y$'; and
\item If there is a surjection $g : X \to Y$, then $X$ has `at least as many elements as $Y$'.
\end{itemize}

We are now in a position to prove this, at least when $X$ and $Y$ are finite. The following theorem is a generalisation of \Cref{thmJectionsAndSizeOfNaturalNumbers}.

\begin{theorem}
\label{thmFiniteSetsAndJections}
Let $X$ and $Y$ be sets.
\begin{enumerate}[(a)]
\item If $Y$ is finite and there is an injection $f : X \to Y$, then $X$ is finite and $|X| \le |Y|$.
\item If $X$ is finite and there is a surjection $f : X \to Y$, then $Y$ is finite and $|X| \ge |Y|$.
\item If one of $X$ or $Y$ is finite and there is a bijection $f : X \to Y$, then $X$ and $Y$ are both finite and $|X| = |Y|$.
\end{enumerate}
\end{theorem}

\begin{cproof}[of (a)]
\item We prove by induction that, for all $n \in \mathbb{N}$, if $Y$ is a finite set of size $n$ and there is an injection $f : X \to Y$, then $X$ is finite and $|X| \le n$.

\begin{itemize}
\item (\textbf{Base case}) Let $Y$ be a finite set of size $0$---that is, $Y$ is empty. Suppose there is an injection $f : X \to Y$. If $X$ is inhabited, then there exists an element $a \in X$, so that $f(a) \in Y$. This contradicts emptiness of $Y$, so that $X$ must be empty. Hence $|X| = 0 \le 0$, as required.

\item (\textbf{Induction step}) Fix $n \in \mathbb{N}$ and assume that, if $Z$ is a finite set of size $n$ and there is an injection $f : X \to Z$, then $X$ is finite and $|X| \le n$.

Fix a finite set $Y$ of size $n+1$ and an injection $f : X \to Y$. We need to prove that $X$ is finite and $|X| \le n+1$.

If $X$ is empty, then $|X| = 0 \le n+1$ as required. So assume that $X$ is inhabited, and fix an element $a \in X$.

By \Cref{lemRemoveElementFromDomainAndCodomain}, there is an injection $f^{\vee} : X \setminus \{ a \} \to Y \setminus \{ f(a) \}$. Moreover, by \Cref{exAddRemoveElementsOfFiniteSets}, $Y \setminus \{ f(a) \}$ is finite and $|Y \setminus \{ f(a) \}| = (n+1) - 1 = n$.

Applying the induction hypothesis with $Z = Y \setminus \{ f(a) \}$ tells us that, $X \setminus \{ a \}$ is finite and $|X \setminus \{ a \}| \le n$. But $|X \setminus \{ a \}| = |X| - 1$ by \Cref{exAddRemoveElementsOfFiniteSets}, and so $|X| \le n+1$, as required.
\end{itemize}

The result now follows by induction.
\end{cproof}

\begin{exercise}
Prove parts (b) and (c) of \Cref{thmFiniteSetsAndJections}.
\end{exercise}

\Cref{thmFiniteSetsAndJections} suggests the following strategies for comparing the sizes of finite sets:

\begin{strategy}[Comparing the sizes of finite sets]
\label{strComparingSizesOfFiniteSets}
Let $X$ and $Y$ be finite sets.
\begin{enumerate}[(a)]
\item In order to prove that $|X| \le |Y|$, it suffices to find an injection $X \to Y$.
\item In order to prove that $|X| \ge |Y|$, it suffices to find a surjection $X \to Y$.
\item \label{strBijectiveProof} In order to prove that $|X| = |Y|$, it suffices to find a bijection $X \to Y$.
\end{enumerate}
Strategy \ref{strBijectiveProof} is commonly known as \textbf{bijective proof}.
\end{strategy}

\subsection*{Closure properties of finite sets}

We now use \Cref{strComparingSizesOfFiniteSets} to prove some \textit{closure properties} of finite sets---that is, operations we can perform on finite sets to ensure that the result remains finite.

\begin{exercise}
Let $X$ be a finite set. Prove that every subset $U \subseteq X$ is finite and $|U| \le |X|$.
\hintlabel{exSubsetOfFiniteSetIsFinite}{%
Given $U \subseteq X$, find an injection $U \to X$ and apply \Cref{thmFiniteSetsAndJections}(a).
}
\end{exercise}

\begin{exercise}
Let $X$ and $Y$ be finite sets. Prove that $X \cap Y$ is finite.
\hintlabel{exIntersectionOfFiniteSetsIsFinite}{%
Recall that $X \cap Y \subseteq X$ for all sets $X$ and $Y$.
}
\end{exercise}

\begin{proposition}
\label{propUnionOfFiniteSetsIsFinite}
Let $X$ and $Y$ be finite sets. Then $X \cup Y$ is finite, and moreover
\[ |X \cup Y| = |X| + |Y| - |X \cap Y| \]
\end{proposition}

\begin{cproof}
We will prove this in the case when $X$ and $Y$ are disjoint. The general case, when $X$ and $Y$ are not assumed to be disjoint, will be \Cref{exSizeOfUnion}.

Let $m = |X|$ and $n=|Y|$, and let $f : [m] \to X$ and $g : [n] \to Y$ be bijections.

Since $X$ and $Y$ are disjoint, we have $X \cap Y = \varnothing$. Define $h : [m+n] \to X \cup Y$ as follows; given $k \in [m+n]$, let
\[ h(k) = \begin{cases} f(k) & \text{if } k < m \\ g(k-m) & \text{if } k \ge m \end{cases} \]
Note that $h$ is well-defined: the cases $k < m$ and $k \ge m$ are mutually exclusive, they cover all possible cases, and $k-m \in [n]$ for all $m \le k < n$ so that $g(k-m)$ is defined. It is then straightforward to check that $h$ has an inverse $h^{-1} : X \cup Y \to [m+n]$ defined by
\[ h^{-1}(z) = \begin{cases} f^{-1}(z) & \text{if } z \in X \\ g^{-1}(z)+m & \text{if } z \in Y \end{cases} \]
Well-definedness of $h^{-1}$ relies fundamentally on the assumption that $X \cap Y = \varnothing$, as this is what ensures that the cases $x \in X$ and $x \in Y$ are mutually exclusive.

%% BEGIN EXTRACT (xtrConclusionExampleTwo) %%
Hence $|X \cup Y| = m+n = |X| + |Y|$, which is as required since $|X \cap Y| = 0$.
%% END EXTRACT %%
\end{cproof}

\begin{exercise}
\label{exSizeOfUnion}
The following steps complete the proof of \Cref{propUnionOfFiniteSetsIsFinite}:
\begin{enumerate}[(a)]
\item Given sets $A$ and $B$, prove that the sets $A \times \{ 0 \}$ and $B \times \{ 1 \}$ are disjoint, and find bijections $A \to A \times \{ 0 \}$ and $B \to B \times \{ 1 \}$. Write $A \sqcup B$\nindex{unionDisjoint}{$\sqcup$}{disjoint union} \inlatex{sqcup}\lindexmmc{sqcup}{$\sqcup$} to denote the set $(A \times \{ 0 \}) \cup (B \times \{ 1 \})$. The set $A \sqcup B$ is called the \textbf{disjoint union}\index{disjoint union} of $A$ and $B$.
\item Prove that, if $A$ and $B$ are finite then $A \sqcup B$ is finite and
\[ |A \sqcup B| = |A| + |B| \]
\item Let $X$ and $Y$ be sets. Find a bijection
\[ (X \cup Y) \sqcup (X \cap Y) \to X \sqcup Y \]
\item Complete the proof of \Cref{propUnionOfFiniteSetsIsFinite}---that is, prove that if $X$ and $Y$ are finite sets, not necessarily disjoint, then $X \cup Y$ is finite and
\[ |X \cup Y| = |X| + |Y| - |X \cap Y| \]
\end{enumerate}
\end{exercise}

\begin{exercise}
Let $X$ be a finite set and let $U \subseteq X$. Prove that $X \setminus U$ is finite, and moreover $|X \setminus U| = |X| - |U|$. How does this result change if $U$ is not required to be a subset of $X$?
\hintlabel{exSizeOfRelativeComplement}{%
Apply \Cref{propUnionOfFiniteSetsIsFinite} with $Y=U$.
}
\end{exercise}

\begin{exercise}
\label{exSizeOfCartesianProduct}
Let $m,n \in \mathbb{N}$. Prove that $|[m] \times [n]| = mn$.
\hintlabel{exBijectionBetweenProductOfBracketSets}{%
Prove by induction on $n \in \mathbb{N}$ that, for all $m,n \in \mathbb{N}$, there is a bijection $[mn] \to [m] \times [n]$.
}
\end{exercise}

\begin{proposition}
\label{propProductOfFiniteSetsIsFinite}
Let $X$ and $Y$ be finite sets. Then $X \times Y$ is finite, and moreover
\[ |X \times Y| = |X| \cdot |Y| \]
\end{proposition}
\begin{cproof}
Let $X$ and $Y$ be finite sets, let $m=|X|$ and $n=|Y|$, and let $f : [m] \to X$ and $g : [n] \to Y$ be bijections. Define a function $h : [m] \times [n] \to X \times Y$ by
\[ h(k,\ell) = (f(k),g(\ell)) \]
for each $k \in [m]$ and $\ell \in [n]$. It is easy to see that this is a bijection, with inverse defined by
\[ h^{-1}(x,y) = (f^{-1}(x), g^{-1}(y)) \]
for all $x \in X$ and $y \in Y$. By \Cref{exBijectionBetweenProductOfBracketSets} there is a bijection $p : [mn] \to [m] \times [n]$, and by \Cref{exCompositeOfBijectionsIsBijection} the composite $h \circ p : [mn] \to X \times Y$ is a bijection. Hence $|X \times Y| = mn$.
\end{cproof}

In summary, we have proved that the property of finiteness is preserved by taking subsets, pairwise unions, pairwise intersections, pairwise cartesian products, and relative complements. 

\subsection*{Infinite sets}

We conclude this section by proving that not all sets are finite---specifically, we'll prove that $\mathbb{N}$ is infinite. \textit{Intuitively} this seems extremely easy: of \textit{course} $\mathbb{N}$ is infinite! But in mathematical practice, this isn't good enough: we need to use our definition of `infinite' to prove that $\mathbb{N}$ is infinite. Namely, we need to prove that there is no bijection $[n] \to \mathbb{N}$ for any $n \in \mathbb{N}$. We will use \Cref{lemFinSetGreatestElement} below in our proof.

\begin{lemma}
\label{lemFinSetGreatestElement}
Every inhabited finite set of natural numbers has a greatest element.
\end{lemma}
\begin{cproof}
We'll prove by induction on $n \ge 1$ that every subset $U \subseteq \mathbb{N}$ of size $n$ has a greatest element.
\begin{itemize}
\item (\textbf{Base case}) Take $U \subseteq \mathbb{N}$ with $|U|=1$. then $U = \{ m \}$ for some $m \in \mathbb{N}$. Since $m$ is the only element of $U$, it is certainly the greatest element of $U$!

\item (\textbf{Induction step}) Fix $n \ge 1$ and suppose that every set of natural numbers of size $n$ has a greatest element.

Let $U \subseteq \mathbb{N}$ with $|U| = n+1$. We wish to show that $U$ has a greatest element.

Since $|U| = n+1 > 0$, we know that $U$ is inhabited, so there is an element $a \in U$. By \Cref{exAddRemoveElementsOfFiniteSets}(a) we have $|U \setminus \{ a \}| = |U| - 1 = n$, and so by the induction hypothesis, $U \setminus \{ a \}$ has a greatest element $m \in U \setminus \{ a \}$. Now either $a < m$ or $a > m$.
\begin{itemize}
\item If $a < m$, then $m$ is the greatest element of $U$; and
\item If $a > m$, then $a$ is the greatest element of $U$.
\end{itemize}
In any case, we see that $U$ has a greatest element.
\end{itemize}
\end{cproof}

\Cref{lemFinSetGreatestElement} is equivalent to the assertion that every inhabited subset of $\mathbb{N}$ with no greatest element is infinite. We can therefore deduce that $\mathbb{N}$ is infinite very easily, as we do in the next theorem.

\begin{theorem}
\label{thmNIsInfinite}
The set $\mathbb{N}$ is infinite.
\end{theorem}

\begin{cproof}
%% BEGIN EXTRACT (xtrContradictionExample) %%
We proceed by contradiction. Suppose $\mathbb{N}$ is finite. Then $|\mathbb{N}| = n$ for some $n \in \mathbb{N}$, and hence $\mathbb{N}$ is either empty (nonsense, since $0 \in \mathbb{N}$) or, by \Cref{lemFinSetGreatestElement}, it has a greatest element $g$. But $g+1 \in \mathbb{N}$ since every natural number has a successor, and $g+1 > g$, so this contradicts maximality of $g$. Hence $\mathbb{N}$ is infinite.
%% END EXTRACT %%
\end{cproof}